\documentclass[letterpaper,11pt,twocolumn]{article}
\usepackage[T1]{fontenc}
\usepackage[utf8]{inputenc}
\usepackage{lmodern}
\usepackage{amsmath}
\usepackage{amsfonts}
\usepackage{amssymb}
\usepackage{amsthm}
\usepackage{graphicx}
\usepackage{color}
\usepackage{xcolor}
\usepackage{url}
\usepackage{textcomp}
\usepackage{listings}
\usepackage{hyperref}
\usepackage{parskip}
\usepackage{biblatex}
\addbibresource{overall.bib}

\title{behaviorism on AI}
\author{Paolo Marzolo}
\date{\today}

\begin{document}

\maketitle
%\tableofcontents

\begin{abstract}
    Skinner, in Verbal Behavior\cite{skinner_verbal_1992}
\end{abstract}

\section{Behaviorism}
To start this, we will need to define the terms we need to use

is behaviorism an expressive enough framework to describe and predict AI agents' results?
to verify it, first we need to show how the basic terminology applies to AI agents. then, we will use the rich history of criticisms that have been raised against the discipline to battle-test our "implementation" of behaviorism. finally, we review why: to avoid mentalistic explanations of behavior and to have a "middle-ground theory" under which comparisons between results are fair...er.


\printbibliography


\end{document}