\documentclass[letterpaper,11pt,twocolumn]{article}
\usepackage[T1]{fontenc}
\usepackage[utf8]{inputenc}
\usepackage{lmodern}
\usepackage{amsmath}
\usepackage{amsfonts}
\usepackage{amssymb}
\usepackage{amsthm}
\usepackage{float}
\usepackage{graphicx}
\usepackage{caption}
\usepackage{subcaption}
\usepackage{color}
\usepackage{xcolor}
\usepackage{url}
\usepackage{textcomp}
\usepackage{listings}
\usepackage{hyperref}
\usepackage{parskip}
\usepackage{todonotes}
\usepackage[
backend=biber,
style=alphabetic,
]{biblatex}
\usepackage{csquotes}
% \setlength{\marginparwidth}{2cm}
\usepackage[inline]{enumitem}
\newlist{inlinelist}{enumerate*}{1}
\setlist[inlinelist]{label=(\roman*)}

\addbibresource{philo.bib}

\title{Machine Behavior: Artificial Intelligence and Radical Behaviorism}
\author{Paolo Marzolo}
\date{\today}

\begin{document}
\twocolumn[
    \begin{@twocolumnfalse}
        \maketitle
        %\tableofcontents
        \begin{abstract}
            Artificial Intelligence powered machines are being increasingly relied upon in all facets of human life. As these machines and their ecosystems grow more complex, equally powerful systems for interpreting, analyzing and predicting their behavior are needed. This raises the need for new, interdisciplinary research to expand upon computer science incorporating insights from other sciences. Here, I explore the philosophical background of Behavioral Science as formulated by B.F. Skinner, and the insights it can give practitioners of Artificial Intelligence when considered in relation to machine behavior.
        \end{abstract}
    \end{@twocolumnfalse}
    \medskip
]

\section*{Introduction}
Artificial Intelligence algorithms have never been more ubiquitous. This prevalence amplifies the potential effect on humanity, but their extreme diversity coupled with the complexity of individual Artificial Intelligence agents makes them opaque to analysis: the functional processes that tie the inputs they receive to the outputs they produce are sometimes hard to interpret even by the scientists who produced them \footnote{The models' size is a core feature: not only is progress in Deep Learning strongly reliant on it\cite{thompsonComputationalLimitsDeep2022}, but the CEO of OpenAI (the company behind ChatGPT) Sam Altman proposed the "amount of compute" as a possible starting point for legislative licensing efforts.}.

In the eloquent words of one of the historical papers on the subject, \enquote{if you open them up and peer inside, all you can see is a big pile of goo}\cite{mozerUsingRelevanceReduce1989a}.
In \cite{rahwanMachineBehaviour2019}, the authors identify \enquote{the emergence of an interdisciplinary field of scientific study}, which studies machine behavior empirically, considering not only the machine itself but also the environment it operates in. The authors also note that \enquote{the scientists who most commonly study the behaviour of machines are the computer scientists, roboticists and engineers who have created the machines in the first place. These scientists may be expert mathematicians and engineers; however, they are typically not trained behaviourists}.

But this is not the first time Artificial Intelligence and Behaviorism have interfaced each other: the start of the field of Artificial Intelligence was grounded in symbols, rules and symbol manipulation, tightly bound with Cognitive Science. The two disciplines were at complete odds with each other\cite{skinnerCognitiveScienceBehaviourism1985}. What, then, will come of trying to reconcile them? In this article, I focus on what the philosophy of Artificial Intelligence stands to gain from Behaviorism. To do so, I give an introduction to the field of Behavior Analysis. After reviewing the main characteristics of Behavior Analysis, I will consider its philosophical background, identified with Radical Behaviorism. This article will introduce Artificial Intelligence practitioners to an alternative, well-defined conceptual framework to the ill-defined, generalist understanding of the relationship between the brain, mind and behavior. Such a conceptual framework does not need to be the only possible one; yet if we are to advance in the discipline of machine behavior, the AI scientist must familiarize himself with it. In particular, I will focus on two factors that distinguished Behavior Analysis from the previous approach to psychological science: the role of the organism (or the agent) in explaining behavior and the impact of uncritically adopting "ordinary language" words in such explanations. Because the first one is one of the roots of its philosophy and the second one is the source of much contention with sciences in similar fields, I will situate them in their environment\footnote{This is meant to avoid mischaracterizing Behaviorism and to also provide an introduction to the conceptual framework}.

\section*{Behaviorism}
The term "Behaviorism" was coined by J. Watson in 1913\cite{maloneDidJohnWatson2014}, as he incorporated earlier studies on reflexes by Pavlov in a general theory of behavior. Watson argued that psychology needed to focus on behavior, instead on the study of mind and consciousness using introspective methods. Watson was also the first prominent psychologist to argue for psychology as a natural science; Radical Behaviorism, the philosophy of that science, later supported this position. Around the same time, Thorndike conducted similar experiments. These studies formed the basis of the experimental analysis of behavior, further formalized in Skinner's doctoral thesis\cite{skinnerBehaviorOrganismsExperimental1999}. Ultimately, this experimental analysis led to a more practically oriented branch, Applied Behavior Analysis, which has been applied successfully in many behavior domains\cite{wlABABehaviorScience2022}.
Thus, three distinct branches were formed:
\begin{description}
    \item[Behavior Analysis] The experiment based, laboratory-focused discipline which discovered and studies the basic principles behind organisms' behavior.
    \item[Applied Behavior Analysis] The discipline aims to apply the basic principles behind behavior discovered in laboratory-based experiments to real-world, practical applications, to \enquote{support positive change in socially important behaviors using basic learning principles}.
    \item[Radical Behaviorism] The philosophy of the Science of Behavior, or Behavior Analysis, is called Radical Behaviorism. It was developed by Skinner\cite{schneiderHistoryTermRadical1987}.
\end{description}

I will now briefly overview of the first two, then consider how the philosophy affects the science direction and why such a perspective is relevant for AI research.

\subsection*{Behavior Analysis}
Pavlov's experiments identified \textit{classical conditioning}: the process in which a specific stimulus (e.g. a sound) is presented with a stimulus (e.g. food) that already elicits a response (salivation) to pair the response with the first stimulus\cite{iClassicalConditioning2023}. The latter formalization by Skinner distinguished between classical conditioning and what he called operant conditioning, a learning process in which modifications to \textit{operants} are elicited by specific consequences. Operants are behaviors that \textit{operate} on the environment, while consequences are respectively called reinforcers or punishments if they increase or decrease the probability of reoccurrence of the operant. This forms the fundamental "three-term contingency": \textit{discriminative stimuli} signal the availability of \textit{reinforcement or punishment} based on a \textit{behavioral response}. Behavior Analysis (more precisely: Skinner) uses these basic principles to explain behavior: actions are behavior, speech is verbal behavior (and to be analyzed as such), thoughts are covert behavior (\enquote{Thinking is behaving (page 66)}\cite{skinnerBehaviorism1976}), social interactions are organisms behaving and providing consequences to each other.

\subsection*{Applied Behavior Analysis}
Applied Behavior Analysis takes Behavior Analysis out of the laboratory and focuses on achieving change in the real world:
\begin{displayquote}
    Applied behavior analysis, or ABA, is a scientific approach for discovering environmental variables that reliably influence socially significant behavior and for developing a technology of behavior change that takes practical advantage of those discoveries.
    \cite{cooperAppliedBehaviorAnalysis2020}
\end{displayquote}
% The same book notes how the seven\footnote{text} dimensions given by Baer and colleagues in 1987\cite{baerSTILLCURRENTDIMENSIONSAPPLIED1987} are still \enquote{useful and relevant signposts} to identify ABA research.
Although the specific operations of this branch are out of this article's scope, it is relevant to note that the philosophical background of ABA directly influences ABA practitioners.

\subsection*{Radical Behaviorism}
As stated, Radical Behaviorism is the philosophy of the science of behavior. It is the foundation against which both Experimental Behavior Analysis and Applied Behavior Analysis measure themselves, and what gives direction to the entire field\footnote{Radical Behaviorism is not the only possible Behaviorism, nor is it the most recent. Nonetheless, it is one of the most studied, and it refers to a single author (Skinner), so it is our object of study here.}. The basic tenet of Behaviorism is that \enquote{A science of behavior is possible}(page 3, \cite{baumUnderstandingBehaviorismBehavior2017}). In addition, it should be considered a \textit{natural} science\cite{baumWhatRadicalBehaviorism2011}\cite{chiesaRadicalBehaviorismPhilosophy1994}(page 179). This results in a relevant distinction from previous psychological approaches: behavioral event do not need any \textit{agent}, as they are not done, but \enquote{are to be explained by other natural events}\cite{baumWhatRadicalBehaviorism2011}.
A recent review\cite{araibaCurrentDiversificationBehaviorism2020} of Behaviorism currents used the approach to the agent problem as the main difference between them.

Then, as natural events, should they be explained in terms of biology, chemistry, and physics? It is a possibility: Skinner himself recognized the usefulness of neuroscience, as \enquote{...independent information about the second link [neuroscience] would obviously permit us to predict the third [behavior] without recourse to the first [history of interactions with the environment]. It would be a preferred type of variable because it would be non-historic}\cite{skinnerScienceHumanBehavior1953}.
Still, such explanations would be outside the scope of a science of behavior: Skinner noted that \enquote{in a science of behavior [...] all statements about the nervous system are theories in the sense that they are not expressed in the same terms and could not be confirmed with the same methods of observation as the facts for which they are said to account}\cite{skinnerAreTheoriesLearning1950}\footnote{I suggest \cite{zilioWhoWhatWhen2016} for a complete account of Skinner's position towards neuroscience as an explanatory system for behavior.}.

Another distinguishing feature of radical behaviorism is its \textit{relational} approach, as identified by Chiesa: \enquote{it seeks to describe (explain) how persons and environments interact, the effect persons have in producing consequences in their environment, and the effect the environment has in shaping and maintaining behavioral repertoires. (page 202)}\cite{chiesaRadicalBehaviorismPhilosophy1994}; this relational approach goes beyond the "mechanistic" view that pushes disciplines to the analysis of separate, discrete parts and their mechanism for interaction\footnote{A similar point is made by Capra\cite{capraTurningPointScience1983}, who argued for the need of a non-dualistic, non-mechanistic psychology, but he considers behaviorism as a mechanistic view.}: \enquote{radical behaviorism dispenses with force or agency, replacing cause with a change in the independent variable and effect with a change in the dependent variable. Behavior (the person) stands in a dependent variable relation to environmental events as independent variables.}(page 122, \cite{chiesaRadicalBehaviorismPhilosophy1994})

The non-dualistic approach requires a stance on inner events, whether they are to be considered stimuli or behavior, like \textit{pain}. Baum \cite{baumUnderstandingBehaviorismBehavior2017} clearly defines the different positions on it:
\begin{quote}
    To methodological behaviorism, pain is an inner private state. To Skinner, pain is a private event or stimulus that results in public pain-behavior. To Ryle, pain is the label of the category “pain-behavior.” To Rachlin, pain is pain-behavior itself. [...]  behavior never originates in private events. (page 53-54) \footnote{Multiple positions were included to provide "upper and lower bounds" to the radical behaviorist's definition.}
\end{quote}

To Skinner, pain (e.g. toothache) is a private event, which may be made public (e.g. by finding a cavity). Where does this leave non-measurable mental causes and explanations? We all use \textit{mental} terms to explain our behavior: \enquote{I couldn't sleep because I was so worried}, \enquote{I felt like doing it}, and so on. Any of such \textit{mental} explanations, thoughts, feelings, sensations, emotions, hallucinations, presuppose the existence of a \textit{mind}. To a radical behaviorist, such a notion is \textit{fictional}: \enquote{Fictional things and events are unobservable, even in principle. No one has observed a mind, urge, impulse, or personality; they are all inferred from behavior. A person who behaves aggressively, for example, is said to have an aggressive personality. No one will ever see the personality, though; one sees only the behavior.} (page 36, \cite{baumUnderstandingBehaviorismBehavior2017}). Unavailability to direct observation, though, is not enough: do atoms exist? Radical behaviorists take issue not with their existence but their explanatory power. Baum shows two types of failure for these \textit{explanatory fictions}: autonomy and superfluity.

Autonomy is identified as the \enquote{ability to behave} (page 37, \cite{baumUnderstandingBehaviorismBehavior2017}). Baum makes a clear distinction: \enquote{No problem arises with assigning behavior to whole organisms; a problem arises when behavior is assigned to parts of organisms, particularly hidden parts}. He notes that autonomy causes explanatory failure when it is assigned to a part of the organism, like what we do when saying "I felt like doing it". This derives from the distinction between "the outside" and "the inner me", the "real self". When we assign behavior to this "real self", we lose all explanatory power, and make the task harder: now what needs to be understood isn't just the measurable, observable behavior, but the hidden, unmeasurable self.

But what if we do not assign autonomy to the inner self? "I did it on impulse", removes all autonomy, but remains completely \textit{superfluous}: this is implied by the derivation of the impulse or inner self, as the reason why it was mentioned as "explanation" in the first place is that it was \textit{inferred} from the behavior. So this inner self, which is determined to exist as it is inferred from the behavior, is then used as explanation for the behavior, forming a circular explanation which provides no insight. Behaviorism's answer to the mind-body problem is that the premise of "how does the mind cause behavior?", which is to say that the mind must exist, is an unfair and unscientific assertion. This makes the mind-body problem a \textit{pseudo-question}.

Behaviorism is not just against a mentalistic account, but, more precisely, against any dualistic view. The future of understanding behavior, to a radical behaviorist, will look like this: the science of behavior will be able to explain human behavior; in the future, if it happens at all, the physiology of the brain will be understood well enough to explain behavior in physiological terms (complex behavior); \enquote{His [the physiologist of the future's] account will be an important advance over a behavioral analysis, because the latter is necessarily “historical”—that is to say, it is confined to functional relations showing temporal gaps.} ("Physiology" chapter, \cite{skinnerBehaviorism1976}\footnote{Skinner goes on to say \enquote{What he discovers cannot invalidate the laws of a science of behavior, but it will make the picture of human action more nearly complete.}}).

\section*{Language and Behaviorism}
By now, it is clear that the radical behaviorist's perspective and explanation of behavior is much different from the way we usually explain it in ordinary language, and as such requires a drastic change in the language used to explore it\footnote{I will not go into detail about definitions here. Additional sources can be found in \cite{chiesaRadicalBehaviorismPhilosophy1994}, p.26, or the glossary of \cite{cooperAppliedBehaviorAnalysis2019}}. Linguistic precision has been a focus of behaviorism since the very start: part of Skinner's dissertation was focused on unintentional implications the word "reflex" had acquired during its history. To understand why ordinary language terms can cause confusion when adopted blindly, consider when we define ourselves as "happy". Different people, who would all identify themselves as "happy", have
\begin{inlinelist}
    \item different responses to the same situation,
    \item no universally agreed upon definition,
    \item different behaviors they would identify as "happy".
\end{inlinelist}
Definitions are not the only reason why behaviorist were careful with the relation between ordinary language and science: Chiesa\cite{chiesaRadicalBehaviorismPhilosophy1994} identifies three problematic features of ordinary language when adopted by scientific psychology:
\begin{description}
    \item[Conceptual inheritance.] \enquote{The important objection to the vernacular in the description of behavior is that many of its terms imply conceptual schemes. I do not mean that a science of behavior is to dispense with a conceptual scheme but that it must not take over without careful consideration the schemes which underlie popular speech}(\cite{skinnerBehaviorOrganismsExperimental1938} p.7). Two examples are noted\footnote{Of course, since the subject is natural language, absolute statements are rare.}.
        \begin{inlinelist}
            \item "Mind" in language: in (English) everyday language, we say "I will bear it in mind", and "It has been on my mind". While these uses have no issue in everyday discourse, once they are transported to the laboratory they drive the scientist to \enquote{try to search for the mind we bear things in or the mind that is put at rest, or the mind we have had something on, we run into the logical problem of trying to submit to scientific analysis a term that has no physical or spatial referent}. Skinner's proposal is to find alternative translations; examples for the above are "I will remember this in the future" and "I have been thinking about this a lot" (a statement on behavior), or "This has been worrying me" (a statement on feelings). This avoids implying the metaphysical dimension that the mind would exist in.
            \item Learning in language: a similar conceptual system is imported through the language of learning; we say "what have you learned?" as if information is stored in the organism and at some later stage retrieved. This drives the scientist to a stimulus-organism-response account of the facts, but Skinner argued that \enquote{Organisms do not acquire behavior as a kind of possession; they simply come to behave in various ways. The behavior is not in them at any time. We say that it is emitted, but only as light is emitted from a hot filament; there is no light in the filament}(\cite{skinnerCognitiveScienceBehaviourism1985}, p.295).
        \end{inlinelist}
    \item[Sentence structure.] Regarding grammatical constraints, Chiesa draws two concerns from Hineline\cite{hinelineLanguageBehaviorAnalysis1980} and Whorf\cite{whorfLanguageThoughtReality1956}. First, he highlights how in English the traditional difference between "doing" words (representing transient events, best expressed as verbs) and "thing" words (semi-permanent, nouns) is not well maintained. \enquote{People remember, think, talk, see, hear, and feel-all verbs. When these actions are transformed into nouns-memory, thought, language, sensation, emotion-as is common in the English language, then scientists are encouraged to look for the things denoted by the nouns.}(\cite{chiesaRadicalBehaviorismPhilosophy1994}, p.35). Then he notes that English requires an agent when expressing action: \enquote{because English language patterns require agents for actions, it is a matter of linguistic rather than logical necessity to supply an agent. In the case of behavior, agency is often ascribed to the organism itself. Behavior does not simply occur; the organism is taken to be the initiating agent}. In the radical behaviorist view, even saying that the environment is the \textit{initiating} agent would be incorrect; Chiesa proposes the passive "Behavior is selected by the environment" as the   \enquote{somewhat better expression}.
    \item[Directional modes.] Lastly, Chiesa notes (once again, quoting Hineline) how usually people interpret personal behavior by relating it to their environment, while the interpretation of others is driven by events or characteristics \textit{inside} the person. In behavioristic accounts, all explanations have the same directionality, violating the cultural norm and possibly causing discomfort in the reader.
\end{description}
In this section, natural language was shown to exert a strong influence on the conceptual background when adopted uncritically; the Behaviorist community strongly advises against such practice, as it can not only fail to explain the phenomena in question but also lead the scientist to a futile chase of the metaphysical inside the real.


\section*{What this is to Artificial Intelligence}
As highlighted so far, Behavior Analysis, Radical Behaviorism and Applied Behavior Analysis form a tightly linked system of, respectively, a science, the philosophy of the science and the real-world applied discipline. Artificial Intelligence is in a very different position: the first noticeable distinction is that, although he was not alone in this endeavor, Skinner is a strong reference point for all three parts of Behavior Analysis. His vision, undoubtedly shaped by the thinkers of his time and the many contributors, still served as a common background to be progressively refined. The philosophy of Artificial Intelligence, instead, was developed by scholars with very different backgrounds: in the "Report of the National Academy Research Briefing Panel on Cognitive Science and Artificial Intelligence"\cite{estesReportNationalAcademy1983}, the authors argue that \enquote{at the core [of both disciplines] is the common aim of understanding intelligence and intelligent behavior}. Such an association is emblematic of the relationship they had: \enquote{the science of intelligence}, a science to be developed as a joint effort, was to explain intelligence in terms of symbols, rules and the relationship between them: \enquote{A key insight in the search for ways to get a handle on the general problem of intelligence is that all intelligent function can be viewed as computation and that computation is a form of symbol processing}. In 1985 Skinner reviewed the cited report, and accused cognitive scientists of \enquote{relaxing standards of definition and logical thinking and releasing a flood of speculation characteristic of metaphysics, literature, and daily intercourse, perhaps suitable enough for such purposes but inimical to science}\footnote{I will not get into the diatribe; suffice it to know, after this publication the two frameworks were like oil and water}. More recently, as the effectiveness of Artificial Intelligence systems increased, the view of AI as the science of intelligence was progressively superseded by AI as an engineering discipline, focused on creating tools and models that solve problems which usually require intelligence.

There are three reasons why machine behavior must now be studied in a dedicated scientific discipline, as highlighted in \cite{rahwanMachineBehaviour2019}:
\begin{inlinelist}
    \item Artificial Intelligence algorithms are ubiquitous; some of its tasks are ranking and selecting the information citizens see, selecting loan recipients, choosing the price of goods, both regulating and exploiting financial markets, dispatching local policing, moving citizens in autonomous vehicles and connecting people in romantic relationships.
    \item AI is complex and opaque; the real, functional relationship between inputs and outputs is extremely complex, and the field of AI explainability is continuously rushing to catch up to the ever-increasing size of models.
    \item When used extensively, any tool has the potential for beneficial or detrimental effect on humanity; but the more complex the tool the harder it is to predict such effect.
\end{inlinelist}

\subsection*{The agent-free view and AI}
As was shown, Radical Behaviorism adopts a strict agent-free view for explaining human behavior. This is consistent with the main features of Radical Behaviorism: it considers the science of behavior a natural science, so events should be explained in terms of other natural events\footnote{As noted earlier, to do so in terms of chemistry, biology and neuroscience is a different discipline, equally justified.}; the object of the study should be behavior, avoiding a dualistic mind-body approach; behavior should be understood as relational, functional interaction of organisms and environment, replacing cause and effect with dependent variable (behavior) and independent variables (environment). Using such an approach for the humans involved in a human-machine interaction would enable both a machine behaviorist and a human behaviorist to frame and explain the interaction between man and machine in scientific terms. Could a similar approach be used to analyze artificial, machine behavior as well? In \cite{lanovazCharacteristicsArgumentsFavor2022}, the author argues for the introduction of an additional dimension in Behavior Analysis dedicated to machine behavior, aptly named Machine Behavior Analysis. In such an analysis, human behavior may become the independent variable: \enquote{how does a machine respond to the changes in the environment produced by the human experimenter?} He goes on to propose that such a science could mirror existing Behavior Analysis by considering different algorithms as different species, while differences in different "runs" of the algorithm could still lead to replicable results: \enquote{Behavior analysts are already aware of this issue because each species, as well as each individual within a species, is unique. For individuals within species, variations in responding may be explained by variations in the initial conditions. These initial conditions include organism-specific genes and prior contact of the individual organism with the environment over which the experimenter has no control}.

\subsection*{Language}
Language was shown to be a critical part of the Behaviorist research program: \enquote{words are the medium through which behavioral scientists express relations; they are the "calculus" of behavioral science}(\cite{chiesaRadicalBehaviorismPhilosophy1994}, p.28). Behaviorists, although they use ordinary language words ("reinforcement", "punishment", "behavior"), critically analyze which words to include in their explanations, as adopting them uncritically leads to unwanted conceptual inheritance. Behavioral explanations also have unusual sentence structure and directional mode, as the organism or anything inside it cannot be identified as the cause of behavior. A similarly critical approach is necessary, as machine behaviorists must strive to avoid such pitfalls; this is especially true for mathematical explanations becoming functional relationships expressed by language. An example of the value of such an approach is given by the previously cited article\cite{lanovazCharacteristicsArgumentsFavor2022}; when considering how a science of machine behavior would adhere to some of the philosophical underpinnings of Behaviorism, he notes:
\begin{quotation}
    Parsimony may prevent the development of unnecessary concepts to explain machine behavior. For example, assume that a machine is learning to greet someone online to help them with a problem. An observer notices that over time the machine selects greetings in a manner that optimizes the time that the person spends online. A parsimonious explanation may be that the machine selects its greeting based on its prior experience in similar situations, which have been associated with interactions of longer durations. A nonparsimonious explanation would be that the machine has developed self-awareness, which leads it to select an appropriate greeting. The latter concept is less parsimonious as it requires more assumptions (e.g., the existence of self-awareness) than the initial explanation that relies exclusively on the observable environment.
\end{quotation}

\section*{Conclusion}
The widespread adoption and ever-increasing complexity of Artificial Intelligence algorithms requires the creation of a new discipline, dedicated to machine behavior, machine-human interaction and machine-machine interaction, that will integrate computer science and artificial intelligence with insights from across the sciences. In this article, I considered Behavior Analysis, and what its philosophical background can provide to Artificial Intelligence researchers, focusing on the agent-free premise of Radical Behaviorism and its approach to explaining behavior. Artificial Intelligence researchers, as they develop and diversify into AI engineers and machine behavior experts, will need to analyze machine behavior not only as a single machine, but also in relationship to other machines and society. This raises both the need and the opportunity for Artificial Intelligence researchers to understand machine behavior and human behavior and the relationship flourishing between them.

\printbibliography


\end{document}